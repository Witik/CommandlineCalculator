\documentclass[11pt,a4paper]{article}
\usepackage[utf8]{inputenc}
%\usepackage[T1]{fontenc}
\usepackage{graphicx}
\usepackage{longtable}
\usepackage{float}
\usepackage{wrapfig}
\usepackage{rotating}
\usepackage{enumitem}
\usepackage[normalem]{ulem}
\usepackage{amsmath}
\usepackage{textcomp}
\usepackage{marvosym}
\usepackage{wasysym}
\usepackage{amssymb}
\usepackage{hyperref}
\usepackage{listings}
\usepackage{verbatim}
\tolerance=1000
\bibliographystyle{plain}
\usepackage{microtype}
\usepackage{tikz}
\usepackage{circuitikz}
\usetikzlibrary{tikzmark,decorations.pathmorphing}
\author{Sasja Gillissen, Martin Huijben, Martijn Terpstra}
\date{\today}
\title{Testing Techniques\\
  \textbf{Assignment 3}}
\hypersetup{
 pdfauthor={Sasja Gillissen, Martin Huijben, Martijn Terpstra},
 pdflang={English}}
\begin{document}
\maketitle

%\tableofcontents


%% This is the second Model-Based Testing (MBT) assignment of the
%% course Testing Techniques, which continues the previous assignment.
%% The purpose of this assignment is to apply Model-Based Testing
%% (MBT) to your System Under Test (SUT) using a second MBT tool, and
%% to compare. You can freely select any MBT tool; below some ideas
%% are given. Try to reuse as much as possible from what you did in
%% the second assignment, i.e., the test architecture and (the
%% structure of) the model, though you will have to express your model
%% in another syntax, of course.

\section{Model-Based Testing}
For this assignment we continued with the test cases used in the previous assignments

\begin{center}
\begin{tabular}{llr}
Category & Input & Expected Output\\
\hline
Exiting & \texttt{exit} & Nothing\\
Function Assignment & \texttt{f(x):= x + 1 ; f(7)} & \texttt{8}\\
Builtin Functions (1) & \texttt{sqrt(ln(log(100)))} & \texttt{0.83}\\
Builtin Functions (2) & \texttt{floor(0.5) + ceil(0.5) + round(0.5)} & \texttt{2}\\
Variable Assignment & \texttt{k = 3 ; k} & \texttt{3}\\
Builtin Variables & \texttt{e * pi} & \texttt{8.5397342226735656}\\
Expression (1) & \texttt{(((0 + 1) * 2) - 3) / 4} & \texttt{-0.2500000000000000}\\
Expression (2) & \texttt{1 / 0} & Error\\
Equality (1) & \texttt{3 = 3} & \texttt{0}\\
Equality (2) & \texttt{2 = 1} & \texttt{1}\\
Equality (3) & \texttt{2 = 3} & \texttt{-1}\\
\end{tabular}
\end{center}



\subsection{MBT Tool Selection}
%% Search for model-based testing tools, select an MBT tool, and give
%% some arguments why you selected that tool. (See the list of MBT
%% tools below.)

For our tool we have chosen OSMO MBT
TOOL\footnote{\url{https://github.com/mukatee/osmo}}.
This tool seemed to be the most appropriate from the list of tools
given in the assignment desription.
Firstly the tool is free software under the LGPL License. This allows
us to freely use this software. Many existing tools are proprietary,
requiring us to buy a license for continued use.
Secondly the chosen tool, and its tests are written in Java. Our SUT
is also written in Java so integration is easier than it would have
been if our SUT and Testing Tool were written in different languages.

\subsection{MBT Modeling}
%% Make a model for your SUT, or part of your SUT, in the input
%% language of the selected MBT tool. In order to allow comparison, it
%% is easiest if you model the same part as you used in
%%the second assignment (first MBT assignment). Explain your model.

Our model is basically the same as the TorXakis model. The biggest difference is that here we can easily work with float values, where in TorXakis this was too troublesome to implement. This means we can actually test the build in variables within epxressions and test the built-in functions. All built-in functions use floats.

When our model starts, it repeatedly chooses an action from the following list and tests it.
\begin{description}[labelindent=1cm]
	\item[\textbf{Expression}] Expressions can be generated out of any combination of +,-,* and numbers. It is checked that the SUT returns the right solution to this expression.
	\item[\textbf{1/0}] OSMO checks if the SUT really gives an error when presented with this input.
	\item[\textbf{equality}] Our model generates two expressions and asks the SUT if the first is greater, equal or smaller than the second. It then checks if the answer given back is correct.
	\item[\textbf{exit}] This tests that giving the 'exit' commando to the SUT returns 'bye!' and that the SUT does not respond to input afterwards.
	\item[\textbf{pi}] Checks if the definition of pi is 3.141592653589793
	\item[\textbf{e}] Checks if the definition of e is 2.718281828459045
	\item[\textbf{function definition}] Generates a random function and remembers it.
	\item[\textbf{built-in functions}] The built-in functions are checked with random input.
	\item[\textbf{function application}] There are two options. Either the function has already been defined or not. If the function is defined, OSMO checks if the SUT gives the right answer. Otherwise it is checked that the SUT gives the corresponding error.
	\item[\textbf{variable definition}] Generates a random variable definition and remembers it.
	\item[\textbf{variable application}] Checks if the SUT gives the right answer if the variable has already been defined. Otherwise it checks if the SUT gives the corresponding error.
\end{description}



\subsection{MBT Test Environment}
%% Adapt the test environment of the previous assignment, if
%% necessary, for model-based testing with the selected MBT tool.
With OSMO we could easily interact with our system under test.

Unlike the previous assignment we did not have to create a special
interface to test our program.

Instead we add the testing environment to our SUT directly. Our
testing tool, and its model are written in java so we could develop it
the same way the program works.

\subsection{MBT Testing}
%% Use the selected MBT tool to generate tests, and execute them on
%% your SUT. Explain your observations and analyse the test results.

Our testing is done at compile time. When we build our SUT using
\verb|mvn install| our tests are also run and the build fails should
the tests fail.

Many random tests are generated during compile time. For the sake of
brevity I have include a small section of the testing output below.

\begin{verbatim}
MDPW=(1 * (((8 + 2) * 1) + 9))
8 + 2 = 10
10 * 1 = 10
10 + 9 = 19
1 * 19 = 19
added variable MDPW to memory
3 * 4 = 12
3 * 12 = 36
5 * 9 = 45
5 * 45 = 225
36 + 225 = 261
3 + 7 = 10
8 * 9 = 72
10 + 72 = 82
3 * 6 = 18
4 + 18 = 22
82 + 22 = 104
261 - 104 = 157
3 - 7 = -4
9 + 7 = 16
-4 * 16 = -64
7 - 6 = 1
2 + 3 = 5
1 - 5 = -4
-64 * -4 = 256
3 + 5 = 8
9 - 8 = 1
1 * 9 = 9
256 - 9 = 247
[157 = 247] = -1
hUz=((0 + 4) + ((2 + (5 + 7)) * 0))
0 + 4 = 4
5 + 7 = 12
2 + 12 = 14
14 * 0 = 0
4 + 0 = 4
added variable hUz to memory
added function XIf to memory
XIf() = (8 + (((4 * 5) + 3) + ((5 - 1) + (8 - 1))))
\end{verbatim}

In the end we could test all our use cases successfully and even more
rigorously than we did with TorXakis in the previous assignment.


\subsection{Deliverable}
%% Give the models, code, adapters, etc. in such a way that we can run
%% it; provide a ’README’. Be prepared to give a demo.

The code for both the SUT, the wrapper and the TorXakis models is
publicly available at
\url{https://github.com/Witik/CommandlineCalculator}.

To start the proxy service and SUT simply run the Sockserv.jar with as
arguments the port to run on and command that you would otherwise
execute. If you are in the root directory of the git project you can
use the command

\section{Comparison}
%% Compare the selected MBT tool of this assignment with TorXakis;
%% consider the following aspects:
\subsection{implementation relation}
Both tools miss functionality that the other tool has. However, the functionality that OSMO misses is easier to overcome.

For example, describing states and state transitions in OSMO is tedious, but our SUT does not have complicated state transitions.
Another example, OSMO does not have automatic expression building, but this was easy to implement.

The problems in TorXakis are difficulties when extracting input (regexp is just not strong enough), difficulties connecting to the SUT and difficulties working with floats. All three of them heavily limited the ease of implementation.

\subsection{support for test input generation as well as output checking}

Both TorXakis and OSMO allow random input generation. In TorXakis it
is easier to generate input. However, this is not hard in OSMO. In
TorXakis there is builtin functionality to generate input.

While OSMO does not come with such expression building tools we can
and have implemented the generation in java. When generating tests,
one should give OSMO a number on which it generates random steps. This
number can also be used as a generator number for generating random
input.

Then you just do

\verb|Random(number).nextInt(size)| to choose what to do next.

Output checking is easier in OSMO. In TorXakis we had to extend our
test adapter to filter the output to something TorXakis could match,
since the regexp expressions TorXakis offered where not strong enough.

\subsection{support for non-determinism}
Our SUT is fully deterministic. Therefore our model does not need
non-determinism.

\subsection{method of test selection}
Both models allow random test selection, redoing previous random
generated test selection, and walking through the model.
\subsection{modeling notation: its expressiveness and ease of use}

The models written with OSMO are java programs. While the syntax may
be different from TorXakis it was significantly easier to setup
because we were already familiar with java.

\subsection{on-line vs. off-line testing, i.e., on-the-fly vs. batch.}

OSMO does off-line batch testing, just like TorXakis. In OSMO we
define a model and let OSMO compare our model with our SUT.

We do not need anything more complicated. Firstly because our system
is lightweight enough that we can easily have a session running just
for testing. On-line testing would not allow us to test anything we
could not easily test offline. Secondly because our system does not
need to operate in real time. Restarting our system just to test it
does not cause any problems. Lastly we assume our SUT is deterministic
and only depends on user input.


%% Hints - Model-Based Testing Tools There are many MBT tools:
%% academic and commercial (often with 30 days free trial period),
%% using various modeling languages, with abstract models and with
%% models as programs, on-line and off-line, supported and
%% unsupported, serious and less serious, . . .. Have a look at
%% websites like this one:
%% http://robertvbinder.com/open-source-tools-for-model-based-testing
%% or this one:
%% http://mit.bme.hu/~micskeiz/pages/modelbased_testing.html or search
%% on the web, or have a look at this list (underlining is used to
%% indicate a slight preference): AETG, Agatha, Agedis, All4Tec
%% MaTeLo, Autolink, Axini Test Manager, Conformiq Qtronic, Cooper,
%% fMBT, Gast, Gotcha, JTorX, NModel, ParTeG, Phact/The Kit,
%% QuickCheck, Reactis, RT-Tester, SaMsTaG, SeppMed MBTsuite,
%% Smartesting CertifyIt, Spec Explorer, Statemate, STG, TestGen
%% (Stirling), TestGen (INT), TestComposer, TGV, TorX, T-Vec, Uppaal
%% Cover, Uppaal Tron, Tveda, TestOptimal, GraphWalker, Tigris, OSMO,
%% . . .

\end{document}

%%  LocalWords:  OSMO SUT TorXakis
