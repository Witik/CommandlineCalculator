\documentclass[11pt,a4paper]{article}
\usepackage[utf8]{inputenc}
%\usepackage[T1]{fontenc}
\usepackage{graphicx}
\usepackage{longtable}
\usepackage{float}
\usepackage{wrapfig}
\usepackage{rotating}
\usepackage{enumitem}
\usepackage[normalem]{ulem}
\usepackage{amsmath}
\usepackage{textcomp}
\usepackage{marvosym}
\usepackage{wasysym}
\usepackage{amssymb}
\usepackage{hyperref}
\usepackage{listings}
\usepackage{verbatim}
\tolerance=1000
\bibliographystyle{plain}
\usepackage{microtype}
\usepackage{tikz}
\usepackage{circuitikz}
\usetikzlibrary{tikzmark,decorations.pathmorphing}
\author{Sasja Gillissen, Martin Huijben, Martijn Terpstra}
\date{\today}
\title{Testing Techniques\\
  \textbf{Assignment 4: Active Learning}}
\hypersetup{
 pdfauthor={Sasja Gillissen, Martin Huijben, Martijn Terpstra},
 pdflang={English}}
\begin{document}
\maketitle

%% In this assignment you will learn models of two systems: one with an userin-
%% terface, such that it is easy to see what it does, and one without user interface.
%% 1
\section{Chocolate bar machine}
%% You will learn a model of an SUT (a chocolate bar machine), by providing
%% counterexamples to hypotheses yourself. Furthermore, you will learn the same
%% SUT by configuring appropriate testing methods to find counterexamples for
%% the learned models automatically.
\subsection{Install learning setup}
%% See the instructions from the tutorial on BlackBoard.
\subsection{Learning methods & counterexamples}
%% 1. Open the chocolate bar machine website in your browser and click around
%% to see how it works.

%% 2. Learn some models (you do not need to reach the complete model) with
%% the learning methods L*, RivestShapire and TTT by providing counterex-
%% amples yourself. What differences do you notice between the learning
%% methods?

%% 3. Choose one learning method to learn the complete model. Give all the
%% counterexamples that you used. Also include all the hypotheses.

%% 4. Why do you think you have learned the complete model? Give an argu-
%% ment.

%% 5. Start learning again with TTT. How many times can you provide the last
%% counterexample you used in learning the complete model, as a counterex-
%% ample in this run?

%% 6. Explain, based on the theory from the lecture, why it is possible that an
%% input trace is accepted multiple times as a counterexample.
\subseciton{1.3}
%% Learn fully automatically

%% 1. Choose some learning method (and write this down).

%% 2. Learn a model automatically with the testing methods RandomWalk,
%% WMethod, and WpMethod. For each of the testing methods, answer
%% the following questions:
%% (a) After how much time does the learner stop?
%% (b) Is the found model the same model as you learned by supplying
%% counterexamples yourself? Why (not)?

%% 3. Explain which testing method worked the best according to you.
\section{Bounded Retransmission Protocol}
%% The Bounded Retransmission Protocol [1] has been developed by Philips to sup-
%% port infrared communication between a remote control and a television. There
%% is a reference implementation of the sender of the protocol. Your task is to find
%% out which out of six implementations built by other manufacturers conform to
%% the reference implementation.
%% 1. Download the jars of the reference implementations and the implementa-
%% tions from the manufacturers from BlackBoard.

%% 2. Learn a model of them all. To do this the learner must communicate via
%% a socket (use a sul of type SocketSul). Get your IP by using
%% InetAddress.getLoopbackAddress(), and reset the system by sending
%% the string “reset”. The input symbols are: IACK, IREQ 0 0 0, IREQ 0 0 1,
%% IREQ 0 1 0, IREQ 0 1 1, IREQ 1 0 0, IREQ 1 0 1, IREQ 1 1 0,
%% IREQ 1 1 1, ISENDFRAME, ITIMEOUT. These symbols are entered
%% with a newline. When starting a jar (by entering java -jar filename.jar
%% in a terminal), it tells you on which port it communicates. After starting a
%% jar, you can start the learner. The output symbols OCONF 0, OCONF 1,
%% and OCONF 2 should all appear in the final model. For these SUTs, Ran-
%% domWalk finds counterexample the quickest, when choosing appropriate
%% values for BasicLearner.randomWalk chanceOfResetting and
%% BasicLearner.randomWalk numberOfSymbols .

%% 3. Download the zip brpcompare.zip and put the package of java code in
%% your project.

%% 4. To find out which of the implementations of the manufacturers is equal to
%% the reference implementation, call the main method of BRPCompare.java
%% on the .dot file of the model of the reference implementation and the
%% model of a implementation of a manufacturer.

%% 5. For each of the implementations of the manufacturers, write down whether
%% they were equal, and if not, provide the counterexample.

%% References
%% [1] Helmink, Leen, Martin Paul Alexander Sellink, and Frits W. Vaandrager.
%% “Proof-checking a data link protocol.” International Workshop on Types for
%% Proofs and Programs. Springer Berlin Heidelberg, 1993.
%% 2

\end{document}
